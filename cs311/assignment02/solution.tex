\documentclass{article}

\usepackage{fancyhdr}
\usepackage{lastpage}
\usepackage{extramarks}
\usepackage[usenames,dvipsnames]{color}
\usepackage{amsmath}
\usepackage{amsthm}
\usepackage{amsfonts}
\usepackage{changepage}
\usepackage{lineno}
\usepackage{algpseudocode}

\topmargin=-0.45in
\evensidemargin=0in
\oddsidemargin=0in
\textwidth=6.5in
\textheight=9.0in
\headsep=0.25in

\linespread{1.1}

\pagestyle{fancy}
\lhead{\hmwkAuthorName}
\chead{\hmwkClass\ (\hmwkClassInstructor\ \hmwkClassTime): \hmwkTitle}
\rhead{\firstxmark}
\lfoot{\lastxmark}
\cfoot{}
\renewcommand\headrulewidth{0.4pt}
\renewcommand\footrulewidth{0.4pt}

\setlength\parindent{0pt}

\newcommand{\enterProblemHeader}[1]{
    \nobreak\extramarks{}{Problem \arabic{#1} continued on next page\ldots}\nobreak{}
    \nobreak\extramarks{Problem \arabic{#1} (continued)}{Problem \arabic{#1} continued on next page\ldots}\nobreak{}
}

\newcommand{\exitProblemHeader}[1]{
    \nobreak\extramarks{Problem \arabic{#1} (continued)}{Problem \arabic{#1} continued on next page\ldots}\nobreak{}
    \stepcounter{#1}
    \nobreak\extramarks{Problem \arabic{#1}}{}\nobreak{}
}

\setcounter{secnumdepth}{0}
\newcounter{homeworkProblemCounter}
\setcounter{homeworkProblemCounter}{1}
\nobreak\extramarks{Problem \arabic{homeworkProblemCounter}}{}\nobreak{}

\newenvironment{homeworkProblem}{
    \section{Problem \arabic{homeworkProblemCounter}}
    \enterProblemHeader{homeworkProblemCounter}
}{
    \exitProblemHeader{homeworkProblemCounter}
}

\newcommand{\hmwkTitle}{Homework\ \#2}
\newcommand{\hmwkDueDate}{September 20, 2013 at 4:30pm}
\newcommand{\hmwkClass}{CS311}
\newcommand{\hmwkClassTime}{Section 3}
\newcommand{\hmwkClassInstructor}{Professor Lathrop}
\newcommand{\hmwkAuthorName}{Josh Davis}

\title{
    \vspace{2in}
    \textmd{\textbf{\hmwkClass:\ \hmwkTitle}}\\
    \normalsize\vspace{0.1in}\small{Due\ on\ \hmwkDueDate}\\
    \vspace{0.1in}\large{\textit{\hmwkClassInstructor\ \hmwkClassTime}}
    \vspace{3in}
}

\author{\textbf{\hmwkAuthorName}}
\date{}

\begin{document}

\maketitle

\pagebreak

\begin{homeworkProblem}
    Using a loop invariant, prove that the following property holds for
    \(MAX\):

    \[
        \forall{0 \leq i < n},\ list[i] \leq MAX(list)
    \]

    \textbf{Solution}

    \begin{proof}
        To prove the following property holds, we will use the following loop
        invariant:
        \\

        \begin{adjustwidth}{2.5em}{0pt}
            When looping, of all of the elements in the range \(list[0 \hdots
            i]\), \(maxValue\) will contain the the value that is the largest
            element in the range of elements.
            \\
        \end{adjustwidth}

        By proving the loop invariant, it will prove the correctness of
        the function \(MAX(list)\). Thus the property will hold.
        \\

        \textbf{Initialization}

        At the beginning of the loop, \(i = 0\). Since \(maxValue = list[0]\),
        there is only one value in the range of elements. It is trivially true
        that the max value of a range with 1 element is always that element.
        Thus the the loop invariant holds.
        \\

        \textbf{Maintenance}

        During each iteration of the while loop on line 3, the value of
        \(list[i]\) is compared to \(maxValue\). If \(list[i] > maxValue\),
        then \(maxValue = list[i]\).
        \\

        Since this happens each loop of the iteration, everytime \(i\) is
        incremented on line 7, the range of elements consists of \(list[0
        \hdots i]\). Thus since \(i\) is being incremented by one, the range is
        growing by one everytime. Since \(maxValue\) is the largest element in
        the range \(list[0 \hdots i - 1]\), we just need to compare the value
        at \(list[i]\) to see if we need to update \(maxValue\).
        \\

        This is exactly what the loop invariant says, thus it proves the
        maintenance step of the loop invariant.
        \\

        \textbf{Termination}

        When the loop terminates, it terminates when \(i < n\). Since \(i\) is incremented
        every iteration in the while loop on line 7, when \(i = n\) the loop will stop.
        \\

        Since \(n\) is the length of the array, it is easy to tell that when
        \(i = n\), all items in \(list\) have been iterated over and thus
        \(maxValue\) will be set to the maximum value in the whole array from
        lines 4--5.
        \\

        Since this corresponds to the termination step of the loop invariant,
        it is easy to see that the loop invariant holds.
    \end{proof}
\end{homeworkProblem}

\pagebreak

\begin{homeworkProblem}
    Prove that a tree with \(n\) vertices has precisely \(n - 1\) edges.

    \begin{proof}
        To prove that a tree with \(n\) vertices has precisely \(n - 1\) edges, we will use induction.
        \\

        \textbf{Base}

        First consider the simplest possible tree, a tree with
        \(n = 0\) vertices.  Since the tree consists of just one vertex, there
        are no edges to other vertices and a tree cannot have a cycle. The
        number of edges is \(0\) or \(n - 1 = 0\). Thus it holds for the base
        case.
        \\

        \textbf{Step}

        We want to prove that for a tree with \(n + 1\) vertices, the tree will
        also have \((n + 1) - 1\) edges.
        \\

        Consider a tree, \(T_0\) with \(n\) vertices. We can create a new tree,
        \(T\) by taking \(T_0\) and adding one vertex to it. This would give \(n + 1\)
        vertices. Because we'd need to connect the vertice to \(T_0\), the
        number of edges would then be:

        \[
            \mbox{edges} = T + 1
        \]

        We know this because a tree must be connected and have no cycles. By
        assuming the induction hypothesis:

        \[
            \begin{split}
                \mbox{edges} &= T + 1
                \\
                &= (n - 1) + 1
                \\
                &= (n + 1) - 1
            \end{split}
        \]

        Thus we have shown that a tree with \(n + 1\) vertices has exactly \((n
        + 1) - 1\) edges.
    \end{proof}
\end{homeworkProblem}

\pagebreak

\begin{homeworkProblem}
    Prove that \(6n^4 + 2n^3 + n + 12 \in \Theta(n^4)\).

    \begin{proof}
        To prove that \(6n^4 + 2n^3 + n + 12 \in \Theta(n^4)\), we must show
        the following:

        \[
            \exists c_1 \exists c_2 \forall n \geq n_0,\ {c_1 \cdot g(n) \leq
            f(n) \leq c_2 \cdot g(n)}
        \]

        For the first equality, it is easy to see that \(n^4 \leq c_1 \cdot
        (6n^4 + 2n^3 + n + 12)\) even for \(c_1 = 1\) and \(n_0 = 1\).
        \\

        Now we must show the second part of the equality:

        \[
            \begin{split}
                6n^4 + 2n^3 + n + 12 &=
                \\
                &\leq 6n^4 + 2n^4 + n^4 + 12n^4
                \\
                &= 24n^4
                \\
                &\leq c_2 \cdot n^4
            \end{split}
        \]

        when \(c_2 = 24\) and \(n_0 = 1\).
        \\

        Therefore, since we have found our two constants, \(c_1 = 1\) and \(c_2
        = 24\), and constrained \(n_0 = 1\), we have proven the proposition.
        Thus the proof is complete.


    \end{proof}
\end{homeworkProblem}

\begin{homeworkProblem}
    Prove a polynomial of degree \(k\), \(a_kn^k + a_{k - 1}n^{k - 1} + \hdots
    + a_1n^1 + a_0n^0\) is a member of \(\Theta(n^k)\).

    \begin{proof}
    \end{proof}
\end{homeworkProblem}

\begin{homeworkProblem}
    Problem 2--12 from the text.

    \begin{proof}
    \end{proof}
\end{homeworkProblem}

\begin{homeworkProblem}
    Problem 2--20 from the text.

    \begin{proof}
    \end{proof}
\end{homeworkProblem}

\begin{homeworkProblem}
    Problem 2--13 from the text.

    \begin{proof}
    \end{proof}
\end{homeworkProblem}

\begin{homeworkProblem}
    Suppose \(f_1 \in \Theta(g_1)\) and \(f_2 \in \Theta(g_2)\). Prove that
    \(f_1/f_2 \in \Theta(g_1/g_2)\).

    \begin{proof}
    \end{proof}
\end{homeworkProblem}

\begin{homeworkProblem}
    Suppose \(f \in O(g)\).
    \\

    \textbf{Part One}

    Prove that \(f + g \in O(g)\).

    \begin{proof}
    \end{proof}

    \textbf{Part Two}

    Prove that \(O(f + g) = O(g)\).

    \begin{proof}
    \end{proof}
\end{homeworkProblem}

\begin{homeworkProblem}
    Prove or disprove: Big Oh defines an equivalence relation on the set of all
    functions: \(f : \mathbb{N} \to \mathbb{R}\).
    \\

    \begin{proof}
    \end{proof}
\end{homeworkProblem}

\begin{homeworkProblem}
    Puzzle.
\end{homeworkProblem}

\end{document}
