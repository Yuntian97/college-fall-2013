\documentclass{article}

\usepackage{fancyhdr}
\usepackage{lastpage}
\usepackage{extramarks}
\usepackage[usenames,dvipsnames]{color}
\usepackage{amsmath}
\usepackage{amsthm}
\usepackage{amsfonts}
\usepackage{changepage}
\usepackage{lineno}
\usepackage[plain]{algorithm}
\usepackage{algpseudocode}
\usepackage{hyperref}

\topmargin=-0.45in
\evensidemargin=0in
\oddsidemargin=0in
\textwidth=6.5in
\textheight=9.0in
\headsep=0.25in

\linespread{1.1}

\pagestyle{fancy}
\lhead{\hmwkAuthorName}
\chead{\hmwkClass\ (\hmwkClassInstructor\ \hmwkClassTime): \hmwkTitle}
\rhead{\firstxmark}
\lfoot{\lastxmark}
\cfoot{}

\renewcommand\headrulewidth{0.4pt}
\renewcommand\footrulewidth{0.4pt}

\setlength{\floatsep}{100pt}
\renewcommand{\algorithmicrequire}{\textbf{Input:}}
\renewcommand{\algorithmicensure}{\textbf{Output:}}
\algrenewcomment[1]{\hfill // #1}

\setlength\parindent{0pt}

\hypersetup{colorlinks=true}

\newcommand{\enterProblemHeader}[1]{
    \nobreak\extramarks{}{Problem \arabic{#1} continued on next page\ldots}\nobreak{}
    \nobreak\extramarks{Problem \arabic{#1} (continued)}{Problem \arabic{#1} continued on next page\ldots}\nobreak{}
}

\newcommand{\exitProblemHeader}[1]{
    \nobreak\extramarks{Problem \arabic{#1} (continued)}{Problem \arabic{#1} continued on next page\ldots}\nobreak{}
    \stepcounter{#1}
    \nobreak\extramarks{Problem \arabic{#1}}{}\nobreak{}
}

\setcounter{secnumdepth}{0}
\newcounter{homeworkProblemCounter}
\setcounter{homeworkProblemCounter}{1}
\nobreak\extramarks{Problem \arabic{homeworkProblemCounter}}{}\nobreak{}

\newenvironment{homeworkProblem}{
    \section{Problem \arabic{homeworkProblemCounter}}
    \enterProblemHeader{homeworkProblemCounter}
}{
    \exitProblemHeader{homeworkProblemCounter}
}

\newcommand{\hmwkTitle}{Homework\ \#8}
\newcommand{\hmwkDueDate}{December 13, 2013 at 4:30pm}
\newcommand{\hmwkClass}{CS311}
\newcommand{\hmwkClassTime}{Section 3}
\newcommand{\hmwkClassInstructor}{Professor Lathrop}
\newcommand{\hmwkAuthorName}{Josh Davis}
\newcommand{\solution}{{\large \bfseries Solution}}

\title{
    \vspace{2in}
    \textmd{\textbf{\hmwkClass:\ \hmwkTitle}}\\
    \normalsize\vspace{0.1in}\small{Due\ on\ \hmwkDueDate}\\
    \vspace{0.1in}\large{\textit{\hmwkClassInstructor\ \hmwkClassTime}}
    \vspace{3in}
}

\author{\textbf{\hmwkAuthorName}}
\date{}

\newcommand{\alg}[1]{\textsc{\bfseries \footnotesize #1}}

\begin{document}

\maketitle

\pagebreak


\section{Note:} I've already taken \textbf{ComS 331}, so that explains the more
formal treatment of these proofs. I just don't want the formalness to be
mistaken as cheating. =]
\\

Cheers!

Josh

\begin{homeworkProblem}
    Prove or disprove: P \(\subseteq\) NP
    \\

    \solution

    \begin{proof}
        To prove that P \(\subseteq\) NP, lets first look at what it means to
        be in P.
        \\

        A given problem \(A\) is in P if it can be solved/decided in
        polynomial time. A given problem is in NP if it can be verified by
        a verifier in polynomial time.
        \\

        Let \(A\) be a problem in P. This means that \(A\) can be solved by
        an algorithm in polynomial time.
        \\

        Thus given a certificate \(c\) that is a solution to \(A\), it easily follows
        that \(c\) can be verified to be a solution to \(A\) because \(c\) was originally
        found in polynomial time because \(A\) is in P.
        \\

        Thus since \(A\) can be verified in polynomial time by such a verifier,
        it proves that \(A\) is also in NP. This shows that \(P \subseteq
        NP\) and thus concludes the proof.
    \end{proof}
\end{homeworkProblem}

\pagebreak

\begin{homeworkProblem}
    Consider the Hitting Set problem which is defined as:
    \\

    \textsc{\bfseries Hitting Set}
    \begin{algorithm}[]
        \begin{algorithmic}[1]
            \Require A collection \(C\) of subsets of a set \(S\), a positive integer \(k\).

            \Ensure Does \(S\) contain a subset \(S'\) such that \(\left\vert
            S' \right\vert \leq k\) and each subset in \(C\) contains one
            element from \(S'\)?
        \end{algorithmic}
    \end{algorithm}

    \textbf{Part A}

    Prove that Hitting Set \(in\) NP.

    \begin{proof}
        To prove that Hitting set in NP, we must prove that there exists a
        verifier such that given a valid certificate \(c\), it can be verified
        in polynomial time.
        \\

        Let \(V\) be such a verifier for the Hitting Set:
        \\

        \(V = \)`` On input \(\langle \langle S, C, k \rangle, c \rangle\)
        where \(S\) is the set and \(C\) is the collection of subsets and \(k\)
        is a positve integer:
        \begin{enumerate}
            \item Decode the certifcate into a new set \(T\).
            \item Check that \(T\) is a valid subset of \(S\).
            \item Check that \(\left\vert T \right\vert \leq k\).
            \item Check that for every subset in \(C\), there is at least
                one item in \(C\) that is in \(T\).
            \item If all the checks pass, \(accept\), else \(reject\).
        \end{enumerate}
    \end{proof}

    Given the input, all calculates operate polynomially on the length of the
    input, thus the Hitting Set problem is in NP.
    \\

    \textbf{Part B}

    Prove that Hitting Set is hard for NP.

    \begin{proof}
    \end{proof}

    \textbf{Part C}

    Prove that Hitting Set is complete for NP.

    \begin{proof}
        The qualifications for being NP-complete are the following:

        \begin{enumerate}
            \item The problem is in NP.
            \item The problem is in NP-hard.
        \end{enumerate}

        The first part was solved in \textbf{Part A}, and the second part was
        solved in \textbf{Part B}, by definition this means that the Hitting
        Set problem is in NP-complete and we conclude the proof.
    \end{proof}
\end{homeworkProblem}

\pagebreak

\begin{homeworkProblem}
    Show that the following problem is NP-complete.
    \\

    \textsc{\bfseries Dense Subgraph}
    \begin{algorithm}[]
        \begin{algorithmic}[1]
            \Require A graph \(G\) and integers \(k\) and \(y\).

            \Ensure Does \(G\) contain a subgraph with exactly \(k\) vertices
            and at least \(y\) edges?
        \end{algorithmic}
    \end{algorithm}

    \solution

    \begin{proof}
        To prove that the Dense Subgraph problem is in NP-complete, we must
        first show the following:

        \begin{enumerate}
            \item The problem is in NP.
            \item The problem is in NP-hard.
        \end{enumerate}

        \textbf{Part One}

        First let's show that Dense Subgraph is in NP. To prove that Dense
        Subgraph is in NP, we must prove that there exists a verifier such
        that given a valid certificate \(c\), it can be verified
        in polynomial time.
        \\

        Let \(V\) be such a verifier for Dense Subgraph where \(c\) is a valid
        certificate thus a subgraph:
        \\

        \(V = \)`` On input \(\langle \langle G, k, y \rangle, c \rangle\):
        \begin{enumerate}
            \item Decode the certificate \(c\) into a new graph, \(G'\).
            \item Test that all nodes and edges in \(G'\) are in \(G\).
            \item Count up the number of vertices in \(G'\) and check that
                it equals \(k\).
            \item Count up the number of edges in \(G'\) and check that
                it equals \(y\).
            \item If everything checks out, \(accept\), else \(reject\).''
        \end{enumerate}

        The verifier \(V\) thus runs in time less than \(O(V + E)\) where \(V =
        \left\vert G.vertices \right\vert\) and \(E = \left\vert G.edges
        \right\vert\) because \(G'\) is a subgraph of \(G\). Since the verifier
        runs in polynomial time, Dense Subgraph is in NP.

        \pagebreak

        \textbf{Part Two}

        Second we must show that a problem already in NP-complete is
        polynomially reducible to Dense Subgraph.
        \\

        Let us consider the \(VETEX-COVER\) problem, which tells us whether or
        not there is a \(k\)-node vertex cover in the graph. \(VERTEX-COVER\)
        is known to be NP-complete thus we can create a polynomial reduction
        from it to the Dense Subgraph problem.
        \\

        Let \(M\) be a Turing machine that solves the \(VETEX-COVER\) problem.
        We can then reduce \(VETEX-COVER\) to the Dense Subgraph, \(VETEX-COVER
        \leq_p\) Dense Subgraph problem by constructing a new Turing machine
        called \(N\) that is constructed as follows:
        \\

        \(N = \)`` On input string \(\langle G, k, y \rangle\) where \(G\) is a
        graph and \(k\) and \(y\) are integers:
        \begin{enumerate}
            \item Run \(M\) on \(G\) and \(k\), the number of vertices we wish
                to find a dense subgraph for.
            \item Take this new subgraph, \(G'\) and count up the edges.
            \item If the number of edges is at least \(y\), return \(true\),
                else \(reject\).
        \end{enumerate}

        After calling \(M\), this reduction runs in \(O(E)\) time where \(E =
        \left\vert G'.edges \right\vert\).  Thus it is polynomial and thus we
        have proven that Dense subgraph is NP-hard.
        \\

        \textbf{Conclusion}

        Since we have proven both conditions for NP-complete, the problem of
        Dense Subgraph is thus NP-complete and this concludes the proof.
    \end{proof}

    \textbf{Reference}

    \begin{enumerate}
        \item Vertex-Cover details from \textit{Introduction to the Theory of
            Computation}, \(2nd\) edition, by Michael Sipser, pg. 288
    \end{enumerate}
\end{homeworkProblem}

\pagebreak

\begin{homeworkProblem}
    Show that the following problem is NP-complete:
    \\

    \textsc{\bfseries Longest Path}
    \begin{algorithm}[]
        \begin{algorithmic}[1]
            \Require A graph \(G\) and positive integer \(k\).

            \Ensure Does \(G\) contain a path that visits at least \(k\)
            different vertices without visiting any vertex more than once?
        \end{algorithmic}
    \end{algorithm}

    \solution

    Prove that the Longest Path problem is NP-complete.

    \begin{proof}
        To prove that the Longest Path problem is in NP-complete, we must first
        show the following:

        \begin{enumerate}
            \item The problem is in NP.
            \item The problem is in NP-hard.
        \end{enumerate}

        \textbf{Part One}

        First let's show that Longest Path is in NP. To prove that Longest Path
        is in NP, we must prove that there exists a verifier such that given a
        valid certificate \(c\), it can be verified in polynomial time.
        \\

        Let \(V\) be such a verifier for Longest Path where \(c\) is a valid
        certificate thus a list of nodes along the path:
        \\

        \(V = \)`` On input \(\langle \langle G, k \rangle, c \rangle\):
        \begin{enumerate}
            \item Iterate over every node in the certficate \(c\) and check
                that each node is in \(G\).
            \item Count up the number of nodes in \(c\) and check that it
                equals \(k\).
            \item If everything checks out, \(accept\), else \(reject\).''
        \end{enumerate}

        Clearly this runs in polynomial time because it only iterates over the
        list of nodes in the path of size \(k\) twice.

        \pagebreak

        \textbf{Part Two}

        Second we must show that a problem already in NP-complete is
        polynomially reducible to Longest Path.
        \\

        Let us consider the \(HAMPATH\) problem, which tells us whether or
        not there is a path start at \(s\) and ending at \(t\) and going through
        every vertex. \(HAMPATH\) is known to be NP-complete thus we can
        create a polynomial reduction from it to the Longest Path problem.
        \\

        Let \(M\) be a Turing machine that solves the \(HAMPATH\) problem.
        We can then reduce \(HAMPATH\) to the Longest Path, \(HAMPATH
        \leq_p\) Longest PAth problem by constructing a new Turing machine
        called \(N\) that is constructed as follows:
        \\

        \(N = \)`` On input string \(\langle G, k \rangle\) where \(G\) is a
        graph and \(k\) is an integer:
        \begin{enumerate}
            \item For all \(s \in G.vertices\):
                \begin{enumerate}
                        \item For all \(t \in G.vertices\):
                            \begin{enumerate}
                                \item Run \(M\) on \(\langle G, s, t \rangle\)
                                    and store the result as \(p\).
                                \item Count up the nodes in \(p\), if it equals
                                    \(k\) then \(accept\), else continue.
                            \end{enumerate}
                \end{enumerate}
            \item If we have reached this point, we know there can't be a
                Longest path of length \(k\) and \(reject\).
        \end{enumerate}

        After calling \(M\), this reduction runs in \(O(V^2)\) because
        it iterates over all vertices in an embedded for loop.  Thus it is
        polynomial and thus we have proven that Longest Path is NP-hard.
        \\

        \textbf{Conclusion}

        Since we have proven both conditions for NP-complete, the problem of
        Longest Path is thus NP-complete and this concludes the proof.
    \end{proof}

    \textbf{Reference}

    \begin{enumerate}
        \item \(HAMPATH\) details from \textit{Introduction to the Theory of
            Computation}, \(2nd\) edition, by Michael Sipser, pg. 260
    \end{enumerate}
\end{homeworkProblem}

\pagebreak

\begin{homeworkProblem}
    Why doesn't the following algorithm suffice to prove it is in P, since it
    runs in \(O(n)\) time?

    \begin{algorithm}[]
        \begin{algorithmic}[1]
            \Function{PrimalityTesting}{$n$}
                \State{} $composite \gets false$
                \\
                \ForAll{$i \in 2 \ldots n - 1$}
                    \If{$n \bmod i = 0$}
                        \State{} $composite \gets true$
                    \EndIf
                \EndFor
                \\
                \State{} \Return{$composite$}
            \EndFunction{}
        \end{algorithmic}
        \caption{Simple primality test}
    \end{algorithm}

    \solution

    This is a common mistake when first learning about the complexity classes.
    The reason is that \alg{PrimalityTesting($n$)}, actually \textit{doesn't}
    run in polynomial time, or \(O(n)\).
    \\

    The reason for this is because in order for it to be in P, it must run in
    polynomial time with respect to the \textit{length} of \(n\), not the
    value.
    \\

    Thus the actual length of \(n\), let this be \(n_0\), is actually \(n_0 = 1
    + \log_{10} n\) for any positive integer \(n\). Thus in order for
    \alg{PrimalityTesting($n$)} to run in polynomial time, it must run in
    \(O(n_0) = O(\log n)\) time if \(n\) is the number to check.
\end{homeworkProblem}

\end{document}
