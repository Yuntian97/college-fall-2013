\documentclass{article}

\usepackage{fancyhdr}
\usepackage{lastpage}
\usepackage{extramarks}
\usepackage[usenames,dvipsnames]{color}
\usepackage{amsmath}
\usepackage{amsthm}
\usepackage{amsfonts}
\usepackage{changepage}
\usepackage{lineno}
\usepackage{algpseudocode}

\topmargin=-0.45in
\evensidemargin=0in
\oddsidemargin=0in
\textwidth=6.5in
\textheight=9.0in
\headsep=0.25in

\linespread{1.1}

\pagestyle{fancy}
\lhead{\hmwkAuthorName}
\chead{\hmwkClass\ (\hmwkClassInstructor\ \hmwkClassTime): \hmwkTitle}
\rhead{\firstxmark}
\lfoot{\lastxmark}
\cfoot{}
\renewcommand\headrulewidth{0.4pt}
\renewcommand\footrulewidth{0.4pt}

\setlength\parindent{0pt}

\newcommand{\enterProblemHeader}[1]{
    \nobreak\extramarks{}{Problem \arabic{#1} continued on next page\ldots}\nobreak{}
    \nobreak\extramarks{Problem \arabic{#1} (continued)}{Problem \arabic{#1} continued on next page\ldots}\nobreak{}
}

\newcommand{\exitProblemHeader}[1]{
    \nobreak\extramarks{Problem \arabic{#1} (continued)}{Problem \arabic{#1} continued on next page\ldots}\nobreak{}
    \stepcounter{#1}
    \nobreak\extramarks{Problem \arabic{#1}}{}\nobreak{}
}

\setcounter{secnumdepth}{0}
\newcounter{homeworkProblemCounter}
\setcounter{homeworkProblemCounter}{1}
\nobreak\extramarks{Problem \arabic{homeworkProblemCounter}}{}\nobreak{}

\newenvironment{homeworkProblem}{
    \section{Problem \arabic{homeworkProblemCounter}}
    \enterProblemHeader{homeworkProblemCounter}
}{
    \exitProblemHeader{homeworkProblemCounter}
}

\newcommand{\hmwkTitle}{Homework\ \#4}
\newcommand{\hmwkDueDate}{October 11, 2013 at 4:30pm}
\newcommand{\hmwkClass}{CS311}
\newcommand{\hmwkClassTime}{Section 3}
\newcommand{\hmwkClassInstructor}{Professor Lathrop}
\newcommand{\hmwkAuthorName}{Josh Davis}

\title{
    \vspace{2in}
    \textmd{\textbf{\hmwkClass:\ \hmwkTitle}}\\
    \normalsize\vspace{0.1in}\small{Due\ on\ \hmwkDueDate}\\
    \vspace{0.1in}\large{\textit{\hmwkClassInstructor\ \hmwkClassTime}}
    \vspace{3in}
}

\author{\textbf{\hmwkAuthorName}}
\date{}

\newcommand{\alg}[1]{\textsc{\bfseries \footnotesize #1}}

\begin{document}

\maketitle

\pagebreak

\begin{homeworkProblem}
    Consider the binary search tree code given.

    \begin{enumerate}
        \item Prove that the best case running time of \alg{BUILD-BST} is \(\Omega(n \lg (n))\).
        \item Prove that the worst case running time of \alg{BUILD-BST} is \(O(n^2)\).
    \end{enumerate}

    \textbf{Solution}
    \\

    \textbf{Part One}

    \begin{proof}
    \end{proof}

    \textbf{Part Two}

    \begin{proof}
    \end{proof}
\end{homeworkProblem}

\pagebreak

\begin{homeworkProblem}
    Consider an AVL tree where every node in the tree has a balance factor of
    0. Prove using induction that any such AVL tree is also full.
    \\

    \textbf{Solution}

    \begin{proof}
        \textbf{Base}
        \\

        \textbf{Step}
    \end{proof}
\end{homeworkProblem}

\pagebreak

\begin{homeworkProblem}
    Write pseudocode that when given a list of elements sorted in ascending
    order, return a complete BST that contains precisely the elements in the
    input list. The algorithm should run in linear time.
    \\

    \textbf{Solution}

    The algorithm works by taking the median of the list and then setting it as
    the root element in the BST then recurses until the range consists of only
    one element. The pseudocode of the algorithm is below:
    \\

    \begin{linenumbers}[1]
        \renewcommand\linenumberfont{\normalfont\bfseries\small}
        \begin{algorithmic}
            \Function{BuildBST}{$list, start, end$}
                \If{$start = end$}
                    \State{} \Return{} new \Call{Node}{$list[start]$}
                \EndIf{}
                \\
                \State{} $mid \gets start + (end - start) / 2$
                \State{} $root \gets$ new \Call{Node}{$list[mid]$}
                \\
                \State{} $root.left \gets$ \Call{BuildBST}{$list, start, mid - 1$}
                \State{} $root.right \gets$ \Call{BuildBST}{$list, mid + 1, end$}
                \\
                \State{} \Return{$root$}
            \EndFunction{}
        \end{algorithmic}
    \end{linenumbers}
\end{homeworkProblem}

\pagebreak

\begin{homeworkProblem}
    Open addressing problem.
\end{homeworkProblem}

\pagebreak

\begin{homeworkProblem}
    Consider the algorithm given to traverse a rooted binary tree where
    \alg{VISIT} is a constant-time function. Using the master theorem, give a
    tight bound on the running time of \alg{TRAVERSE} as a function of the number
    of nodes in \(t\).
    \\

    \textbf{Solution}
\end{homeworkProblem}

\pagebreak

\begin{homeworkProblem}
    Consider the following recurrence relation:
    \[
        T(n) = \left\{
            \begin{array}{ll}
                1 & \quad n = 1 \\
                7 \cdot T(\dfrac{n}{2}) + 1 & \quad n > 1
            \end{array}
        \right.
    \]

    \begin{enumerate}
        \item Use the master theorem to show that \(T(n) \in \Theta(n^{\lg 7})\).
        \item Use induction to show that \(T(n) = \dfrac{1}{6} \cdot (7 \cdot
              n^{\lg 7} - 1)\). You may assume that \(n\) is a power of two.
    \end{enumerate}

    \textbf{Solution}
    \\

    \textbf{Part One}
    \\

    \textbf{Part Two}
\end{homeworkProblem}

\pagebreak

\begin{homeworkProblem}
    Using the given operations of a priority queue do the following:

    \begin{enumerate}
        \item Provide a pseudocode implementation of the \alg{enqueue\((e)\)} and \alg{dequeue\(()\)} operations of a queue
              that uses a priority queue to store its data.
        \item Provide pseudocode implementations of the \alg{push\((e)\)} and \alg{pop\(()\)}
              operations of a stack that uses a priority queue to store its data.
    \end{enumerate}

    \textbf{Solution}
    \\

    \textbf{Part One}
    \\

    \textbf{Part Two}
\end{homeworkProblem}

\pagebreak

\begin{homeworkProblem}
    Consider the algorithm given for quicksort. Using the knowledge that
    quicksort depends on the behavior of \alg{SELECT-PIVOT}, assume that
    \alg{SELECT-PIVOT} runs in constant time and \alg{PARTITION} runs in time
    \(\Omega(end - start)\), do the following:

    \begin{enumerate}
        \item What \alg{SELECT-PIVOT} behavior will lead to worst case \alg{QUICKSORT} behavior? Use the master
              theorem to derive a tight bound on \alg{QUICKSORT}'s worst case running time.
        \item What \alg{SELECT-PIVOT} behavior will lead to the best case \alg{QUICKSORT} behavior? Use the
              master theorem to derive a tight bound on \alg{QUICKSORT}'s best case running time.
    \end{enumerate}
\end{homeworkProblem}

\pagebreak

\begin{homeworkProblem}
    A 10-foot rope is stretched across a gap. \(n\) ants are distributed on the
    rope and each ant faces either direction. At the same instant, the ants
    begin to walk, moving forward at a rate of 10 feet per minute. If two ants
    collide, they immediately reverse direction. When an ant reaches the end of
    the rope, it steps off the rope and can no longer block the other ants.

    \begin{enumerate}
        \item What is the largest possible amount of time that can elapse
              before the last ant steps off the rope?
        \item In what situation does the worst-case time arise?
    \end{enumerate}
\end{homeworkProblem}

\end{document}
