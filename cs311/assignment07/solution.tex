\documentclass{article}

\usepackage{fancyhdr}
\usepackage{lastpage}
\usepackage{extramarks}
\usepackage[usenames,dvipsnames]{color}
\usepackage{amsmath}
\usepackage{amsthm}
\usepackage{amsfonts}
\usepackage{changepage}
\usepackage{lineno}
\usepackage[plain]{algorithm}
\usepackage{algpseudocode}
\usepackage{tikz}
\usepackage{hyperref}

\usetikzlibrary{arrows}
\usetikzlibrary{positioning}

\topmargin=-0.45in
\evensidemargin=0in
\oddsidemargin=0in
\textwidth=6.5in
\textheight=9.0in
\headsep=0.25in

\linespread{1.1}

\pagestyle{fancy}
\lhead{\hmwkAuthorName}
\chead{\hmwkClass\ (\hmwkClassInstructor\ \hmwkClassTime): \hmwkTitle}
\rhead{\firstxmark}
\lfoot{\lastxmark}
\cfoot{}

\renewcommand\headrulewidth{0.4pt}
\renewcommand\footrulewidth{0.4pt}

\setlength{\floatsep}{100pt}
\renewcommand{\algorithmicrequire}{\textbf{Input:}}
\renewcommand{\algorithmicensure}{\textbf{Output:}}
\algrenewcomment[1]{\hfill // #1}

\setlength\parindent{0pt}

\hypersetup{colorlinks=true}

\newcommand{\enterProblemHeader}[1]{
    \nobreak\extramarks{}{Problem \arabic{#1} continued on next page\ldots}\nobreak{}
    \nobreak\extramarks{Problem \arabic{#1} (continued)}{Problem \arabic{#1} continued on next page\ldots}\nobreak{}
}

\newcommand{\exitProblemHeader}[1]{
    \nobreak\extramarks{Problem \arabic{#1} (continued)}{Problem \arabic{#1} continued on next page\ldots}\nobreak{}
    \stepcounter{#1}
    \nobreak\extramarks{Problem \arabic{#1}}{}\nobreak{}
}

\setcounter{secnumdepth}{0}
\newcounter{homeworkProblemCounter}
\setcounter{homeworkProblemCounter}{1}
\nobreak\extramarks{Problem \arabic{homeworkProblemCounter}}{}\nobreak{}

\newenvironment{homeworkProblem}{
    \section{Problem \arabic{homeworkProblemCounter}}
    \enterProblemHeader{homeworkProblemCounter}
}{
    \exitProblemHeader{homeworkProblemCounter}
}

\newcommand{\hmwkTitle}{Homework\ \#7}
\newcommand{\hmwkDueDate}{December 6, 2013 at 4:30pm}
\newcommand{\hmwkClass}{CS311}
\newcommand{\hmwkClassTime}{Section 3}
\newcommand{\hmwkClassInstructor}{Professor Lathrop}
\newcommand{\hmwkAuthorName}{Josh Davis}
\newcommand{\solution}{{\large \bfseries Solution}}

\title{
    \vspace{2in}
    \textmd{\textbf{\hmwkClass:\ \hmwkTitle}}\\
    \normalsize\vspace{0.1in}\small{Due\ on\ \hmwkDueDate}\\
    \vspace{0.1in}\large{\textit{\hmwkClassInstructor\ \hmwkClassTime}}
    \vspace{3in}
}

\author{\textbf{\hmwkAuthorName}}
\date{}

\newcommand{\alg}[1]{\textsc{\bfseries \footnotesize #1}}

\begin{document}

\maketitle

\pagebreak

\begin{homeworkProblem}
    Use simulated annealing to find approximate solutions to the optimization
    variant of \alg{No-Three-In-Line}.
    \\

    \textbf{Part A}

    Specify in pseudocode a fitness function that evaluates potential solutions
    to \alg{No-Three-In-Line}.
    \\

    \solution

    \begin{algorithm}[]
        \begin{algorithmic}[1]
            \Function{NoThreeInLineFitness}{$n, board$}
                \State{} $cost \gets 0$
                \ForAll{$i \in 1 \ldots n$}
                    \ForAll{$j \in 1 \ldots n$}
                        \If{$null$}
                            \State{} $largest \gets edge$
                        \EndIf
                    \EndFor
                \EndFor
            \EndFunction{}
        \end{algorithmic}
        \caption{Fitness algorithm}
    \end{algorithm}

    \textbf{Part B}

    Specify an appropriate set of moves for the \alg{No-Three-In-Line} problem.
    \\

    \solution
    \\

    \textbf{Part C}

    Discuss strengths and weaknesses of your fitness function and set of moves.
    Are potential solutions that the set of moves label as 'near' actually
    similar in a useful sense? Why or why not?  How does this affect the
    effectiveness of the simulated annealing algorithm?
    \\

    \solution
\end{homeworkProblem}

\pagebreak

\begin{homeworkProblem}
    Backtracking.
    \\

    \textbf{Part A}

    Specify a set of partial candidates and a method for generating candidate
    extension steps.
    \\

    \solution

    We can start with an empty board and try going by one row at a time. If we
    do this, we just need to try to place a dot if it isn't collinear with two
    other dots. If we find a place that doesn't clash, we add it to our
    solution and continue.
    \\

    \textbf{Part B}

    Specify in pseudocode a method that efficiently determines whether a
    partial candidate should be pruned.
    \\

    \solution

    Pruning algorithm that takes in a candidate represented as the board and
    the next position in the board to place a point in, \(i, j\):

    \begin{algorithm}[]
        \begin{algorithmic}[1]
            \Function{ShouldPrune}{$board, i, j$}
                \If{points in this row, \(i\)}
                    \State{} \Return{$false$}
                \EndIf

                \If{points in diagonal from \(i, j\)}
                    \State{} \Return{$false$}
                \EndIf
                \\
                \State{} \Return{$true$}
            \EndFunction{}
        \end{algorithmic}
        \caption{Backtracking pruning function}
    \end{algorithm}

    \textbf{Part C}

    Given answers to (a) and (b), would a backtracking algorithm correctly
    solve the decision variant of \alg{NoThreeInLine} for every \(n\)?
    \\

    \solution

    It should for as long as \(n \geq 2\) and \(n \geq 3\). Other than that,
    the backtracking algorithm will go back to the last place that wrorked if
    it can't find one that does and go with it.
    \\

    \textbf{Reference}

    \begin{enumerate}
        \item Best explanation I could find of the problem: \\

            \url{http://www.geeksforgeeks.org/backtracking-set-3-n-queen-problem/}
    \end{enumerate}
\end{homeworkProblem}

\pagebreak

\begin{homeworkProblem}
    \solution

    \begin{algorithm}[]
        \begin{algorithmic}[1]
            \Function{UnigramSegmentation}{$S, err$}
                \State{} \Return{0}
            \EndFunction{}
        \end{algorithmic}
        \caption{Unigram variant of sequence segmentation}
    \end{algorithm}
\end{homeworkProblem}

\pagebreak

\begin{homeworkProblem}
    \solution

    \begin{algorithm}[]
        \begin{algorithmic}[1]
            \Function{NGramSegementation}{$S, err$}
                \State{} \Return{0}
            \EndFunction{}
        \end{algorithmic}
        \caption{N-gram variant of sequence segmentation}
    \end{algorithm}
\end{homeworkProblem}

\pagebreak

\begin{homeworkProblem}
    \alg{CountBooleanParenthesizations{($e$)}}

    \begin{algorithm}[]
        \begin{algorithmic}[1]
            \Require A sequence called \(e\) of \(T, F, \vee, \wedge, \mathbin{\oplus}\) symbols.

            \Ensure Number of ways to parenthsize \(e\) to be \(true\).
        \end{algorithmic}
    \end{algorithm}

    \solution

    First let \(n\) be the number of values of \(T\) and \(F\) that appear in
    \(e\). This means that there are \(n - 1\) operators in \(e\). Let these
    values be \(v_i\) and the operators be \(o_i\) for \(i = 1 \ldots n\).
    \\

    Now we can create two recurrence functions to calculate our dynamic
    programming solution.
    \\

    Let \alg{True($i, j$)} be the number of ways to parenthsize \(e\) such that
    it evaluates to true and let \alg{False($i, j$)} be the number of ways to
    parenthsize such that it evaluates to false.
    \\

    We can then define \alg{True($i, j$)} and \alg{False($i, j$)} where \(i\)
    is the beginning of the parenthsization and \(j\) is the end as such:

    \[
        T(i, j) = \sum\limits_{k=i}^j \left\{
            \begin{array}{ll}
                (T(i, k) + F(i, k)) + (T(k + 1, j) + F(k + 1, j)) - F(i, k) \cdot F(k + 1, j) & \quad o_k = \vee
                \\
                T(i, k) \cdot T(k + 1, j) & \quad o_k = \wedge
                \\
                T(i, k) \cdot F(k + 1, j) + F(i, k) \cdot T(k + 1, j) & \quad o_k = \mathbin{\oplus}
            \end{array}
        \right.
    \]

    The reasoning is below:

    \begin{enumerate}
        \item \textbf{\(\vee\)}: We find the times each operand can be true by
            totaling up the parenthesizations of each part of the expression
            and just subtracting when \(\vee\) is false, which is only when
            both are false.
        \item \textbf{\(\wedge\)}: The number of true parenthesizations can
            only be the product of the subproblems.
        \item \textbf{\(\mathbin{\oplus}\)}: Since xor is only true when
            neither are true or neither are false, we take the sum of each
            product of the true and false for each subexpression.
    \end{enumerate}

    Similarly as above, we can now determine our \alg{False($i, j$)} by just
    negating \alg{True($i, j$)}:

    \[
        F(i, j) = \sum\limits_{k=i}^j \left\{
            \begin{array}{ll}
                F(i, k) \cdot F(k + 1, j) & \quad o_k = \vee
                \\
                (T(i, k) + F(i, k)) + (T(k + 1, j) + F(k + 1, j)) - T(i, k) \cdot T(k + 1, j) & \quad o_k = \wedge
                \\
                T(i, k) \cdot T(k + 1, j) + F(i, k) \cdot F(k + 1, j) & \quad o_k = \mathbin{\oplus}
            \end{array}
        \right.
    \]

    \textbf{Reference}

    \begin{enumerate}
        \item Dynamic programming solution was greatly inspired by problem 6:
            \url{http://courses.csail.mit.edu/6.006/fall10/handouts/dpproblems-sol.pdf}
        \item I used it to learn more about dynamic programming. It certainly took awhile to
            understand but it certainly helped a lot.
    \end{enumerate}
\end{homeworkProblem}

\pagebreak

\begin{homeworkProblem}
    \alg{EggDroppings($n, h$)}

    \begin{algorithm}[]
        \begin{algorithmic}[1]
            \Require \(n\) the number of durable eggs and \(h\) the height of
            an apartment building

            \Ensure The minimum number of drops necessary to \textit{guarantee}
            that the value of \(c\) will be discovered before you run out of
            eggs.
        \end{algorithmic}
    \end{algorithm}

    \textbf{Part A}

    Let \(n = 1\). What dropping strategy for minimum value of \(c\)? What is
    the largest number of drops your strategy could require, and what value(s)
    of \(c\) cause this situation to arise?
    \\

    \solution
    \\

    \textit{Strategy:} When \(n = 1\), the strategy of starting at the lowest
    floor and dropping it until it breaks. The value of \(c\) will be equal to
    one less than the floor that it broke on.
    \\

    \textit{Largest Number:} The largest number of drops would of course be if we had
    to drop it until we reach the top of the apartment or when \(c = h\).
    \\

    \textbf{Part B}

    Let \(n = 2\). What dropping strategy for minimum value of \(c\)? What is
    the largest number of drops your strategy could require, and what value(s)
    of \(c\) cause this situation to arise?
    \\

    \solution
    \\

    \textit{Strategy:} When \(n = 2\), the strategy of splitting the number of
    floors in half until an egg cracks, then reverting back to the base case of
    testing from the lowest known floor until we find when it cracks.
    \\

    It is very similar to binary search, we have a \(low\) and \(high\) values
    that are initially \(low = 1\) and \(high = h\), then take the mid point.
    If the egg doesn't crack, \(low = mid\) and then we repeat. When the egg
    cracks, we set \(high = mid\) and start at \(low\) and go until \(high\).
    \\

    \textit{Largest Number:} The largest number of drops would be when \(c = \lceil h
    / 2 \rceil \) because we would drop it once at \(\lceil h / 2 \rceil\) and it would break and
    then require \((h / 2) - 1\) additional drops resulting in \(h / 2\) total
    drops.
    \\

    \textbf{Part C}

    Let \(n \in \mathbb{Z}^{+}\). Write a dynamic programming algorithm that
    efficiently solves \alg{EggDroppings}.
    \\

    \solution

    The dynamic programming algorithm is as follows:

    \[
        E(n, h) = \left\{
            \begin{array}{ll}
                0 & \quad \text{if } n = 0, k = 1
                \\
                \min\limits_{i = 1 \ldots n} \{ 1 + \max(E(n, h - i), E(n - 1, i - 1)) \}
            \end{array}
        \right.
    \]

    The reasoning is because it is the smallest over possible max values where
    we check the value of when the egg doesn't break \(E(n, h - i)\), and when
    the egg does break, \(E(n - 1, i - 1)\).

\end{homeworkProblem}

\end{document}
