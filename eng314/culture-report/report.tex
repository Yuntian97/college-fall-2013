\documentclass[11pt,a4paper,twoside]{report}

\usepackage{fancyhdr}
\usepackage{extramarks}
\usepackage{palatino}
\usepackage[margin=2cm]{geometry}
\usepackage{nameref}
\usepackage{setspace}


\pagestyle{fancy}

\topmargin=-0.45in
\evensidemargin=0in
\oddsidemargin=0in
\textwidth=6.5in
\textheight=9.0in

\renewcommand{\chaptermark}[1]{\markboth{#1}{}}
\renewcommand{\sectionmark}[1]{\markright{#1}}
\fancyhead[L]{\nouppercase{\firstleftmark}}
\fancyhead[C]{Disalvo Corp.}
\fancyhead[R]{\nouppercase{\lastrightmark}}
\fancyfoot{}
\fancyfoot[LE,RO]{\thepage}

\renewcommand{\contentsname}{Table of Contents}

\title{
    \vspace{.1in}
    \textmd{\textbf{\Huge The Employee Hitchhiker's Guide to the Cultures}}\\
    \vspace{2in}
    \Large{Employees of Disalvo Corporation: \textit{Don't Panic}}
    \vspace{2in}
}

\author{
    Shawn Hetrick
    \\
    Nick Ruhser
    \\
    Josh Davis
    \vspace{1in}
}

\date{October 29, 2013}

\begin{document}

\thispagestyle{empty}
\renewcommand\headrulewidth{0pt}
\renewcommand\footrulewidth{0pt}

\maketitle

\newpage

\renewcommand\headrulewidth{0.4pt}
\renewcommand\footrulewidth{0.4pt}

\setcounter{secnumdepth}{0}
\setcounter{tocdepth}{3}

\pagenumbering{roman}

{\Large \tableofcontents}

\pagebreak

\doublespacing
\pagenumbering{arabic}

\chapter{Introduction}

The advancements in technology in the last century have enabled a marketplace
beyond what has existed in the past. While previous generations could focus on
a single country and its culture, we need to go beyond our borders.

As members of this company, Disalvo Corporation, we need all our members to
setup to this global marketplace. We need to execute sales, make deals, and
help the world.

The purpose of this report is to enable every single member of our workforce to
travel to a foreign culture and succeed. To succeed, we will present different
cultures from different countries and discuss them. The topics discussed will
include the most essential items for success. The topics are basic
communication, social and work norms, and dining etiquette.

We have searched far and wide to present to you the best possible information.
We have scoured the web, read books, talked to residents, and even visited some
of the countries ourselves to better learn about the countries.

With these skills mastered, you will be in the best possible position
when traveling abroad. Buckle your seat belts, put your tray table up,
and remember, \textit{Don't panic}.

\section{List of Countries}

The countries that we have selected represent the type of global strategy that
we are trying to achieve. We have picked a country from each continent that we
wish to do business. Spanning multiple continents will allow you to get a
glimpse of diversity in the world. The countries that we have selected are
Brazil, Russia, and South Africa.

\chapter{Brazil}

Brazil is a very dynamic culture. The diversity in Brazil stems from its
history of colonization. Originally, Brazil was a Portuguese colony.  The
heavy demand for sugar lead to many Africans being bought and used a slave
labor. The indigenous people were also enslaved and forced to produce sugar
from sugar cane. In 1803, the Portuguese royal family fled to Brazil and
created an independent country (Kagan, Ozment, \& Turner, p. 614). On
October 5, 1988, Brazil drafted and signed a constitution creating a
federal republic (Central Intelligence Agency The World Factbook). Brazil
will continue to be an important part of the world economy and
understanding the culture will lead to successful business ventures.

\section{Communication}

Understanding how to communicate with Brazilians will help you succeed in
your business ventures in Brazil. Portuguese is the most commonly spoken
language of Brazil. Speaking Spanish to Brazilians can be taken as an
insult, as most Brazilians are proud of their native language.  Most of the
upper management of Brazilian companies and educated individuals will be
fluent in English (Herrington, p. 62). However it is most likely a good
idea to have a translator present just to understand side conversations. It
may be a good Idea to bring along your own translator. Translators are very
expensive if you hire one in Brazil. Non-verbal communication is very
important in Brazil and you must be aware of some gestures and what they
mean.

Brazil, like many other cultures, has their own set of non-verbal communication
gestures. The gesture to remember is to never give the OK sign with your thumb
and index finger as it means asshole (Herrington, p. 76). Another gesture that
is inappropriate would be to slam your fist into your palm because it has a
sexual connotation (Herrington, p. 76). Very much like the United States
shrugging your shoulders, shaking your head and nodding all have the same
meanings in Brazil.  Brazilians like to speak with one another at a much closer
distance than you may be used to. This may make you uncomfortable, but if you
back up the person you are talking to will most likely step forward. Typically,
Brazilians will be very physical with one another. They will often touch each
other throughout the conversation. This is not seen as flirtatious, it is just
how they communicate with one another. Understanding the social and work norms
of Brazil is just as important as communication.

\section{Social \& Work Norms}

There are many important things to remember when attending meetings with
Brazilians. Titles are very important in the business culture of Brazil. It
is imperative that you know the titles of the people you are meeting. ““A
company’s chief executive is usually called the Presidente, Diretor
superintendente is equivalent to a managing director or vice president of a
division (Herrington, p. 75)”.  “Anyone with a college degree of any kind
is referred to as Douctor or Doutora (Herrington, p. 75)”. When you first
arrive at the meeting it is a custom to shake everyone’s hand that is
attending the meeting. Knowing and using the titles will help you appear
professional. Dressing for the occasion will make you look professional.

Professional dress in Brazil is very much like the United States. A suit and
tie are very common for men and women (Herrington, p. 81). The one thing you
shouldn’t wear are shorts, they are taboo for work attire and are usually only
found at the beach (Herrington, p. 81). Shoes are important as they are one of
Brazil’s biggest industries (Central Intelligence Agency The World Factbook).
You should ensure that your shoes are in good condition and if possible
Brazilian shoes would be an added bonus. Understanding the social norms of
Brazil will help prepare you for what to experience outside the professional
world.

The beach, soccer, and bars are very integrated into Brazilian culture. The
beaches will tend to be crowded year round (Herrington, p. 86). The Brazilians
created the “dental floss” bathing suit and are very proud of it. Women at the
beach will often wear much less clothing than an American may be used to. The
beach is a great place to socialize and spend the day.  Soccer can be seen as a
religion in Brazil (Herrington, p. 87). When Brazil makes it to the semi-finals
of the World Cup, Brazilian business will send their employees home to watch
the game (Herrington, p. 87). Soccer will continue to be a major influence to
Brazilians and you should familiarize yourself with the sport. The bars in
Brazil are less of a club scene and more of a quiet place to socialize. The
bars are a home away from home for Brazilians (Herrington, p. 87). Brazil, like
most South American countries, runs on the 24 hour clock. See appendix for
table of times.  When you go out to eat in Brazil there are some very important
customs that you should be aware of.

\section{Dining Etiquette}
Brazilian dining etiquette can be very different from the United States.
Nearly everything you will eat will require the use of utensils
(Herrington, p. 83). It isn’t polite to use your hands during dining, so
try to refrain from using them. Lunch is typically the largest meal of the
day (Herrington, p. 83). However, dinner will typically be the longest meal
of the day and will begin around 8 or 9 P.M. and can last to the early
morning hours (Herrington, p. 83). If invited to dinner with your business
partner, it is inappropriate to talk business over the meal and should be
saved till after the meal. Most meals in Brazil will be centered on beef,
pork, or seafood. Brazil has a very diverse spread of meals and will vary
by the region you are visiting. Some meals will include cuts of meat that
Americans typically discard. These meals honor their slave heritage and it
would be considered rude if you refuse these dishes.

Understanding the communication, work norms, social norms, and dining etiquette
are the keys to success on your trip to Brazil. Brazilians are very proud of
their heritage and customs. Following the guidelines provided will enhance your
ability to respectfully interact with Brazilians. Using the information
provided will ensure your success in Brazil.

\chapter{Russia}

Traveling to Russia might be necessary for your position in this company.
Russia has one of the top 10 economies based on gross domestic product (GDP)
[cite]. Being successful on a international business trip means knowing the
country and the culture. This section of the report will focus on Russian
culture. Specifically it will discuss communication, norms (social and work),
and dining etiquette in the business setting.

\section{Communication}

Communicating well with your clients in Russia is of utmost importance. We want
the people you talk with to focus on the content of what you say and not
peculiarities you might have being a foreigner.

A first encounter you have while in Russia is quite different than here in the
United States. Often Russians don’t smile when meeting a new person and
accompany it with a handshake in a business setting.

Regarding the language you should speak, over half of the people in the top
Russian companies can speak English. The number drops from there in other
lesser companies but is still considerably high. Be aware that some people
might translate their Russian into rude or direct English. Don’t be offended by
this; it is merely a communication quirk of translation.

When meeting someone, don’t be surprised to learn that they want to be called
“Pasha” or “Sasha” if their name really is “Pavel” or “Alexander”,
respectively. It is common practice to have nicknames that might not look
anything like their given name.

A rather humorous side-effect of this nicknaming is that some characters can
have many names in Russian literature. The book The Brothers Karamazov by
Fyodor Dostoevsky has a main character named Alexei Karamazov. Throughout the
book, he is referred to by his real name or one of his 8 other nicknames:
Alyosha, Alyoshka, Alyoshenka, Alyoshechka, Alexeichik, Lyosha, Lyoshenka.
Getting use to all the nicknames takes a handful of chapters. [cite]

\section{Social \& Work Norms}

Before talking about the social norms in Russia, it’ll be important to
understand what a quintessential Russian is. A Russian is typically
characterized as being incredibly patient, [page 70].

Americans have a very well established “spatial bubble” when it comes to people
we know and people we meet. In Russia, expect this bubble to be much smaller
and closer to you. Since people will be closer to you their voices will be just
as loud. Don’t be alarmed if it seems at first that a Russian is yelling at
you. If the person is closer to you, it might just be the proximity.

When going to a meeting,  it is important to dress such that you fit in with
the rest of your colleagues. Wearing appropriate clothing is almost as
important as how you greet the person. It is important to wear dark clothing
and a suit. Make sure your tie, shoes, socks, and everything else match the
attire. Wearing too many colors might not be the best and might give off the
impression of laziness.

Humor in Russia, like in all places, will show up in conversations frequently.
Russians are often attributed as being very good at black humor. Black humor is
defined as “the juxtaposition of morbid and farcical elements to give a
disturbing effect.” To give an example of this, there is a book called The
Master and Margarita written in sometime in the early 20th century. It centers
around events in Moscow after Satan shows up with a witch and a black cat that
enjoys drinking vodka and playing chess. It is full of black humor and is
considered by some critics as being the best novels of the 20th century. [cite]

\section{Dining Etiquette}

Be sure to eat until you are stuffed. According to tradition, it is rude to the
cook to refuse food. Toasting is another tradition that is strong in Russia.
You can expect the host to start it. If you are at a loss of things to say, a
few common ones include: “to your health” and “one for the road.”

\chapter{South Africa}

South Africa has rich culture leading back to its colonial days. Dutch settled
on the Cape of the country and British settled on the mainland of South Africa
during this time. Between its colonial days and up until 1994, when the country
was transformed into a democracy, the country had experienced ethnical, racial
conflicts, and even the introduction to slavery, which was abolished by the
British in 1834 (David Coplan). But ever since the democratic transformation in
1994. The country has been freed of ethnic and racial violence amongst
themselves and has been succeeding as a nation ever since. Understanding the
South African culture will help us better understand the people that live
within the country. With this knowledge you should be able to effectively dine,
communicate, and understand social norms in South Africa.

\section{Communication}

There are eleven official languages spoken in South Africa, including English,
Afrikaans, Tamil, Urdu, and the southern Bantu Languages (David Coplan).
According to David Coplan, Afrikaans is still the most generally used in
everyday conversation in South Africa because of family heritage. That being
said, you will be able to find English speakers there and will not need a
translator unless in a non-English speaking area. English is used for
education, law, government, and formal communication in South Africa so your
business colleagues should speak English.


This next piece of advice is a cultural norm very different from American
culture, so it is important to remember. If you are asked out to dinner or to a
social event, pay attention to whether or not they imply that his or her spouse
is also going. If the answer is yes, in the request to go out, it is implied
that the invitation is extended to both your family and their own (David
Coplan). When attending, you should bring your spouse; if not it may be seen as
disrespectful. Again, due to Muslim culture, genders may dine separately. Bring
your spouse will give their spouse someone to eat and converse with.

As mentioned earlier, in South Africa you will be perfectly fine speaking
English, most of the business men and women will be fluent in English. If your
colleagues start speaking Afrikaans be patient with them and ask them to
clarify to you and do not be afraid to ask if you would like to know what they
said to each other. Above all else, be respectful and patient, for some of them
might not be as fluent as others.

\section{Social \& Work Norms}

In South Africa there are a lot of different cultures that were introduced in
the country’s colonization. Having roots that go back to European culture,
South Africa has a long history of being primarily Christian in its background
including small minority groups of Jewish, Muslim, and Hindu within the country
(David Coplan). The main two countries that colonized in South Africa were the
Dutch and the British. This may explain how western culture still resides in of
the country today.


According to David Coplan’s article on South Africa, the majority of people in
South Africa are known to be polite and courteous to others. As you move into
larger cities within the country, they are more likely to express discontent in
situations they feel uncomfortable in. African culture acknowledges pride and
honor to the elder men and women in the communities. If invited into a home,
remember to pronounce their name followed by “son of fathers/mothers name” out
of respect for the parents (David Coplan). South Africa remains similar to
other cultures’ worldwide recognition of family honor.


Overall what David Coplan found while in South Africa is most of the people are
very hospitable, helpful, sympathetic, and anxious to avoid verbal conflict.
Remember to be respectful of other people’s space and beliefs in order to not
cause conflicts. In this country the worst criticism bestowed upon someone is
being called “rude” (David Coplan).  In order to avoid the criticizing title of
“rude”, show respect to all, including extended family; this will lead to a
much smoother time.

\section{Dining Etiquette}

When traveling to South Africa, the cuisine is an aspect of local culture that
should not be ignored. Being a colonial territory South Africa; it has many
different cultures that have influenced the country’s history and its food. The
main influences are Western foods and Asian foods (Mike Lininger, International
Dining Etiquette).  According to Mike Lininger, an expert in South African
culture, reports that “Local food is rich in rice, yams and cassava (a root
vegetable), plus breads, fresh vegetables, and fruits.” Fish is the most common
protein found in South Africa because it is more readily available due to its
close proximity to rivers and the ocean. South Africa is a predominantly Muslim
country, which affects the acceptance of certain foods. For example, eating
pork can be considered offensive.  Alcohol can also be disrespectful, so follow
this rule of thumb; only order alcohol if your host does, this is the best way
to play it safe. Men and women are accustomed to sitting separately in South
Africa because of the Jewish and Muslim history so also be aware if one of your
colleagues is of the opposite sex you do not want to leave them out of your
business conversations.

According to Mike Liniger, Eating styles are similar to European culture, which
permits the use of both hands, however two guide lines should be followed. The
first is that utensils should not switch hands during the meal.  The second is
that your right hand should be used as your dominant hand, while using the left
only if your meal requires it. Similar to custom in western cultures it is
polite to wash ones hands before eating, but it differs in that you should also
wash your hands when you are done eating your meal.

When eating at a restaurant it is very normal to tip the waiter/waitress ten
percent of the meal price. They will refuse and tell you that it isn’t
necessary but really they will greatly appreciate it and it is a common
courtesy. In South Africa it is custom that whoever invites you to lunch will
also be paying for the meal you are eating (Mike Lininger, International Dining
Etiquette). Over lunch it is normal for men to talk about work and family while
they are eating. Be prepared to sit around after the meal to have conversations
which could just be an extension of what you talked about over lunch or more
business. So keep in mind you might spend time after eating continuing your
discussions.

In summary the four most important things to remember are to wash your hands
before and after the meal, eat with your right hand primarily, choose your food
out of respect for the other persons believes, and do not drink alcohol unless
it is offered to you. If you can follow these four basic things you should be
able to survive a business lunch in South Africa.

\chapter{Conclusion}

The purpose of this document is to inform English 314 employees on the cultures
we are sending them to.  The information we provide will allow each employee to
interact with these cultures professionally and successfully. The most
important components you need to know when conducting business in a foreign
culture are communication, work norms, social norms, and dining etiquette of
those cultures. We will layout the key points of each of these for Russia,
South Africa, and Brazil. The information we have gathered comes from web
sites, books, and personal experiences. These sources all provide up to date
accurate information. In the culture is very different from the United
States. \cite{cite1} \cite{cite2}

\appendix

\chapter{Brazil Bibliography}

\bibliographystyle{plain}
\bibliography{brazil}

\chapter{Russia Bibliography}

\bibliographystyle{plain}
\bibliography{russia}

\chapter{South Africa Bibliography}

\bibliographystyle{plain}
\bibliography{south-africa}


\end{document}
