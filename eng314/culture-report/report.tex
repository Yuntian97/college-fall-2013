\documentclass[11pt,a4paper,oneside]{report}

\usepackage{fancyhdr}
\usepackage{extramarks}
\usepackage{palatino}
\usepackage[margin=2cm]{geometry}
\usepackage{nameref}
\usepackage{setspace}


\pagestyle{fancy}

\topmargin=-0.45in
\evensidemargin=0in
\oddsidemargin=0in
\textwidth=6.5in
\textheight=9.0in

\renewcommand{\sectionmark}[1]{\markboth{#1}{}}
\renewcommand{\subsectionmark}[1]{\markright{#1}}
\fancyhead[L]{\nouppercase{\firstleftmark}}
\fancyhead[R]{\nouppercase{\lastrightmark}}

\renewcommand{\contentsname}{Table of Contents}

\title{
    \vspace{.1in}
    \textmd{\textbf{\Huge The Employee Hitchhiker's Guide to the Cultures}}\\
    \vspace{2in}
    \Large{Employees of Disalvo Corporation: \textit{Don't Panic}}
    \vspace{2in}
}

\author{
    Shawn Hetrick
    \\
    Nick Ruhser
    \\
    Josh Davis
    \vspace{1in}
}

\date{October 29, 2013}

\begin{document}

\thispagestyle{empty}
\renewcommand\headrulewidth{0pt}
\renewcommand\footrulewidth{0pt}

\maketitle

\newpage

\renewcommand\headrulewidth{0.4pt}
\renewcommand\footrulewidth{0.4pt}

\setcounter{secnumdepth}{0}
\setcounter{tocdepth}{3}

\pagenumbering{roman}

{\Large \tableofcontents}

\pagebreak

\doublespacing
\pagenumbering{arabic}

\section{Introduction}\label{first}
Intro here..

\section{Countries}\label{first}
Countires!

\subsection{Brazil}\label{second}

Brazil is a very dynamic culture. The diversity in Brazil stems from its
history of colonization. Originally, Brazil was a Portuguese colony.  The
heavy demand for sugar lead to many Africans being bought and used a slave
labor. The indigenous people were also enslaved and forced to produce sugar
from sugar cane. In 1803, the Portuguese royal family fled to Brazil and
created an independent country (Kagan, Ozment, \& Turner, p. 614). On
October 5, 1988, Brazil drafted and signed a constitution creating a
federal republic (Central Intelligence Agency The World Factbook). Brazil
will continue to be an important part of the world economy and
understanding the culture will lead to successful business ventures.

\subsubsection{Communication}\label{third}
Understanding how to communicate with Brazilians will help you succeed in
your business ventures in Brazil. Portuguese is the most commonly spoken
language of Brazil. Speaking Spanish to Brazilians can be taken as an
insult, as most Brazilians are proud of their native language.  Most of the
upper management of Brazilian companies and educated individuals will be
fluent in English (Herrington, p. 62). However it is most likely a good
idea to have a translator present just to understand side conversations. It
may be a good Idea to bring along your own translator. Translators are very
expensive if you hire one in Brazil. Non-verbal communication is very
important in Brazil and you must be aware of some gestures and what they
mean.

Brazil, like many other cultures, has their own set of non-verbal communication
gestures. The gesture to remember is to never give the OK sign with your thumb
and index finger as it means asshole (Herrington, p. 76). Another gesture that
is inappropriate would be to slam your fist into your palm because it has a
sexual connotation (Herrington, p. 76). Very much like the United States
shrugging your shoulders, shaking your head and nodding all have the same
meanings in Brazil.  Brazilians like to speak with one another at a much closer
distance than you may be used to. This may make you uncomfortable, but if you
back up the person you are talking to will most likely step forward. Typically,
Brazilians will be very physical with one another. They will often touch each
other throughout the conversation. This is not seen as flirtatious, it is just
how they communicate with one another. Understanding the social and work norms
of Brazil is just as important as communication.

\subsubsection{Social \& Work Norms}\label{third}
There are many important things to remember when attending meetings with
Brazilians. Titles are very important in the business culture of Brazil. It
is imperative that you know the titles of the people you are meeting. ““A
company’s chief executive is usually called the Presidente, Diretor
superintendente is equivalent to a managing director or vice president of a
division (Herrington, p. 75)”.  “Anyone with a college degree of any kind
is referred to as Douctor or Doutora (Herrington, p. 75)”. When you first
arrive at the meeting it is a custom to shake everyone’s hand that is
attending the meeting. Knowing and using the titles will help you appear
professional. Dressing for the occasion will make you look professional.

Professional dress in Brazil is very much like the United States. A suit and
tie are very common for men and women (Herrington, p. 81). The one thing you
shouldn’t wear are shorts, they are taboo for work attire and are usually only
found at the beach (Herrington, p. 81). Shoes are important as they are one of
Brazil’s biggest industries (Central Intelligence Agency The World Factbook).
You should ensure that your shoes are in good condition and if possible
Brazilian shoes would be an added bonus. Understanding the social norms of
Brazil will help prepare you for what to experience outside the professional
world.

The beach, soccer, and bars are very integrated into Brazilian culture. The
beaches will tend to be crowded year round (Herrington, p. 86). The Brazilians
created the “dental floss” bathing suit and are very proud of it. Women at the
beach will often wear much less clothing than an American may be used to. The
beach is a great place to socialize and spend the day.  Soccer can be seen as a
religion in Brazil (Herrington, p. 87). When Brazil makes it to the semi-finals
of the World Cup, Brazilian business will send their employees home to watch
the game (Herrington, p. 87). Soccer will continue to be a major influence to
Brazilians and you should familiarize yourself with the sport. The bars in
Brazil are less of a club scene and more of a quiet place to socialize. The
bars are a home away from home for Brazilians (Herrington, p. 87). Brazil, like
most South American countries, runs on the 24 hour clock. See appendix for
table of times.  When you go out to eat in Brazil there are some very important
customs that you should be aware of.

\subsubsection{Dining Etiquette}
Brazilian dining etiquette can be very different from the United States.
Nearly everything you will eat will require the use of utensils
(Herrington, p. 83). It isn’t polite to use your hands during dining, so
try to refrain from using them. Lunch is typically the largest meal of the
day (Herrington, p. 83). However, dinner will typically be the longest meal
of the day and will begin around 8 or 9 P.M. and can last to the early
morning hours (Herrington, p. 83). If invited to dinner with your business
partner, it is inappropriate to talk business over the meal and should be
saved till after the meal. Most meals in Brazil will be centered on beef,
pork, or seafood. Brazil has a very diverse spread of meals and will vary
by the region you are visiting. Some meals will include cuts of meat that
Americans typically discard. These meals honor their slave heritage and it
would be considered rude if you refuse these dishes.

Understanding the communication, work norms, social norms, and dining etiquette
are the keys to success on your trip to Brazil. Brazilians are very proud of
their heritage and customs. Following the guidelines provided will enhance your
ability to respectfully interact with Brazilians. Using the information
provided will ensure your success in Brazil.

\subsection{Russia}\label{second}
Russia content.

\subsubsection{Communication}\label{third}
Communication here!

\subsubsection{Social \& Work Norms}\label{third}
Social and work norms here.

\subsubsection{Dining Etiquette}\label{third}
Dining here!

\subsection{South Africa}\label{second}

South Africa content.

\subsubsection{Communication}\label{third}
Communication here!

\subsubsection{Social \& Work Norms}\label{third}
Social and work norms here.

\subsubsection{Dining Etiquette}\label{third}
Dining here!

\section{Conclusion}\label{first}
Conclusion here..

\end{document}
